The coupling force is how the two units affect each other. Knowledge of this state is key for accident prevention, drive-line control, and energy efficiency optimization applications. When the two units push against each other, for example during the braking of a tractor or propulsion of a trailer, a compressing force is achieved. A too-high compressing coupling force can create an unstable system \cite{erdinc_safe_2023} causing accidents such as jackknifing. By early section of an increasing coupling force these types of accidents can usually be prevented. %, much like in a jackknife this is when the angle between the objects grows large and collapse into each other.

% Old version - Updating
Accurately estimating the dynamic states of articulated heavy vehicles, such as tractor-semitrailer combinations, is essential for many automation and safety-critical applications. These states include coupling force, articulation angle, and lateral velocity, which are critical for ensuring safe and efficient operation, and will be the main focus.

In the automotive sector, this angle is called the articulation angle. The articulation angle greatly affects the real-time dynamics of the vehicle and is also required to describe how multiple states of the tractor are represented in the trailer. Since the coupling force is how the two units interact, it must be accurately modeled for each unit.

With knowledge of acceleration (force) over time, we can determine the lateral velocity, which is crucial in understanding each unit's movement through the environment. A high lateral velocity results in tire side-slip, generating lateral force. The lateral force of the tires is reflected in the coupling force, as such it is required to jointly estimate both lateral velocity and coupling force. In addition, a non-zero lateral velocity limits the braking potential.

%Integrated in the text above
Despite the importance of precise knowledge of the coupling forces, limited research has been conducted in this field. To jointly estimate these states there is even less. Currently, no existing method accurately estimates a tractor-semitrailer combination's longitudinal and lateral coupling forces using commonly available measurements on a vehicle. The review paper \cite{habibnejad_korayem_review_2022} surveys numerous studies estimating states of articulated vehicles, but none focused on the coupling force. Previous research mainly focuses on car-trailer combinations, where, in contrast to heavy vehicles, data such as distances between axles, trailer wheel speeds, and static load for each axle cannot be assumed to be readily available. In addition, the dynamics of heavy vehicles are different, for instance, because of multiple axles, a higher center of mass (CoM), and the trailer as the heavier unit. As a result, previously developed methods are not directly applicable to heavy vehicles. The approach in \cite{habibnejad_korayem_estimation_2022}, while applied on heavier units, requires the second-order time derivative of the vehicle yaw rate and its lateral velocity, which are unrealistic to assume available, or at least not with any precision. 

Two closely related studies \cite{cheng_parameter_2011, jeong_estimation_2022} have produced good results for the lateral velocity. However, both studies assumes a small articulation angle, which cannot be guaranteed. The articulation angle is also crucial in determining how the coupling force affects each unit. Small angle assumption is however required to avoid describing an articulated unit as a differential algebraic equation (DAE)\cite{ghandriz_computationally_2020}. Methods to use DAEs in estimators are available \cite{purohit_development_2018 , becerra_applying_2001} but have not been considered here.

Old version:
We develop a method that accurately estimates the stated essential states using sensor observations commonly available on commercial vehicles. To perform these estimations, we propose a novel dynamic model, implemented using an Unscented Kalman Filter (UKF)\cite{julier_new_1997}, that considers the specific opportunities and challenges related to the heavy vehicle application. Specifically, the proposed model jointly describes the coupling forces and the complete motion of the vehicle combination in a closed form by decoupling the models of the two units. The decoupling is performed by modeling the coupling point as a damper, leading to a much simpler dynamic model in which the change in coupling force can be modeled from the other more well-known states. We also advocate for utilizing the uncertainty provided by Kalman Filters to inform of the potential magnitude of the error. 

The proposed method is evaluated using simulated data from a high-fidelity tractor-semitrailer model where the vehicle is driven in different patterns. The simulator uses realistic road data, and sensor observations are corrupted with noise and biases. The main contributions of this paper are:

\begin{enumerate}
   \item \textbf{Introduction of a novel dynamic} model that decouples the tractor and trailer units by modeling the coupling point as a damper, simplifying the estimation of the complete vehicle. Can be used with a standard inference model, jointly estimating critical states.
   % \item \textbf{Joint estimation of critical states} (coupling forces, articulation angle, and lateral velocity) using an Unscented Kalman Filter (UKF), tailored to the specific dynamics of heavy vehicles.
   \item \textbf{Validation through high-fidelity simulations}, demonstrating the method's robustness and effectiveness under various driving conditions, incorporating realistic sensor noise and biases.
   \item \textbf{Advocating for uncertainty quantification}, emphasizing its importance in the heavy vehicles area is often overlooked.
\end{enumerate}
 